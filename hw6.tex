\documentclass[12pt]{article}

\usepackage{amsmath,amssymb,amsthm}
\usepackage{mathtools}
\usepackage{graphicx}
\usepackage[margin=1in]{geometry}
\usepackage{fancyhdr}
\usepackage[export]{adjustbox}

\setlength{\parindent}{0pt}
\setlength{\parskip}{5pt plus 1pt}
\setlength{\headheight}{14.5pt}
\pagestyle{fancyplain}
\lhead{\textbf{\NAME\ (\UID)}}
\chead{\textbf{HW\HWNUM}}
\rhead{CS 6140, \today}

\DeclarePairedDelimiter{\ceil}{\lceil}{\rceil}

\begin{document}\raggedright%

\newcommand\NAME{Elliot Hatch}
\newcommand\UID{u0790511}
\newcommand\HWNUM{6}

\section{Finding q*}

\begin{enumerate}
	\item

		$q_*$ vectors:

		\begin{tabular}{r | r | r | r}
			Matrix Power & State Propegation & Random Walk & Eigen-Analysis \\
			\hline
			0.0358 & 0.0358 & 0.0361 & 0.0358 \\
			0.0572 & 0.0572 & 0.0615 & 0.0572 \\
			0.0581 & 0.0581 & 0.0712 & 0.0581 \\
			0.0792 & 0.0792 & 0.0849 & 0.0792 \\
			0.0858 & 0.0858 & 0.0976 & 0.0858 \\
			0.0660 & 0.0660 & 0.0693 & 0.0660 \\
			0.1579 & 0.1579 & 0.1571 & 0.1579 \\
			0.1716 & 0.1716 & 0.1610 & 0.1716 \\
			0.1373 & 0.1373 & 0.1288 & 0.1373 \\
			0.1510 & 0.1510 & 0.1327 & 0.1510 \\
		\end{tabular}

	\item 
		To determine the needed $t$ I try each $1\leq t \leq 2048$, and compute the error as $sum(abs(q_{*t}-q_{*1024}))$. I stop when the error in the current iteration is greater than the error in the last iteration.

	Matrix Power:\\
	t: 798\\
	error: $2.4187*10^{-13}$

	State Propegation:\\
	t: 798\\
	error: $2.4177*10^{-13}$

	\item

		\begin{enumerate}
			\item Matrix Power:

				Pro: Easy to compute when $t$ is a power of 2.\\
				Con: Inaccuate when $t$ is small.\\

			\item State Propegation:

				Pro: Low memory overhead.\\
				Con: Inaccuate when $t$ is small.

			\item Random Walk:

				Pro: Each step is very easy to compute, low memory overhead.\\
				Con: Results have some variance due to inherent randomness.

			\item Eigen-Analysis:

				Pro: Very accurate, independent of $t$.\\
				Con: Expensive to compute when $M$ has high dimensionality.
		\end{enumerate}

	\item
		The chain is ergotic. Analyzing the structure of the graph we see that it is fully connected and there are no absorbing or transient states---all three of the highly
		connected ``clusters'' in the graph have some edge out to another cluster. We can also see that the chain is aperiodic---although cycles can occur randomly, the graph
		does not force a cyclical walk with a fixed period.

		We can also see that $P^7$ has no non-zero entries, so all $P^n$ with $n \geq 7$ also contain no zero entries.
	
	\item The 4th column vector is $[0,0,0.3,0,0.3,0.4,0,0,0,0]$ so we have:

		Node 3: 0.3\\
		Node 5: 0.3\\
		Node 6: 0.4\\
\end{enumerate}

\end{document}
